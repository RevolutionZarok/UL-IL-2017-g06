\subsection{Operation Model for handlingDelayPassed}

\label{OM-handlingDelayPassed}


The \msrcode{handlingDelayPassed} operation has the following properties:

	\begin{operationmodel}
	\addheading{Operation}
	\adddoublerow{handlingDelayPassed}{used to determine if the crisis stood too longly in a pending status since last reminder.}


	\addrowheading{Return type}
	\addsinglerow{ptBoolean}

		

	\addrowheading{Post-Condition (functional)}
	\addnumberedsinglerow{PostF}{true iff the crisis is in pending status and if the duration between the current ctState clock information and the last reminder is greater than the crisis reminder period duration.}

	\end{operationmodel}



	% ------------------------------------------
	% MCL Listing
	% ------------------------------------------
	\vspace{1cm}
	The listing~\ref{OM-ctCrisis-handlingDelayPassed-MCL-LST} provides the \msrmessir (MCL-oriented) specification of the operation.
	
	\scriptsize
	\vspace{0.5cm}
	\begin{lstlisting}[style=MessirStyle,firstnumber=auto,captionpos=b,caption={\msrmessir (MCL-oriented) specification of the operation \emph{handlingDelayPassed}.},label=OM-ctCrisis-handlingDelayPassed-MCL-LST]

	
	
	/* Post Functional:*/ 
	postF{let TheSystem:ctState in
	let CurrentClockSecondsQty:dtInteger in
	let vpLastReminderSecondsQty:dtInteger in
	let CrisisReminderPeriod:dtSecond in
	if 
	( /* Post F01 */
	  self.rnSystem = TheSystem
	  and self.status = pending
	  and TheSystem.clock.toSecondsQty() = CurrentClockSecondsQty
	  and TheSystem.vpLastReminder.toSecondsQty() = vpLastReminderSecondsQty
	  and TheSystem.crisisReminderPeriod = CrisisReminderPeriod
	  and CurrentClockSecondsQty.sub(vpLastReminderSecondsQty).gt(CrisisReminderPeriod) = true
	)
	then (result = true)
	else (result = false)
	endif}
	
	
	\end{lstlisting}
	\normalsize 
	
	
	
	





